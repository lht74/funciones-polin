
\documentclass[12pt]{exam}
\usepackage[spanish]{babel}
\usepackage{amsmath} % Para notación matemática
\usepackage{multicol} % Para crear columnas
\usepackage[left=10mm, right= 20mm, bottom= 20mm, top= 35mm, headsep=10mm, headheight=0mm]{geometry}
\usepackage{setspace}  
\usepackage{graphicx}
\renewcommand\partlabel{\thepartno)}
%%% DEFINICIONES PARA MODIFICAR APARIENCIA DEL ENCABEZADO%%%
\usepackage{etoolbox}
\makeatletter
\patchcmd{\@fullhead}{\hrule}{\hrule\vskip2pt\hrule height 2pt}{}{}
\patchcmd{\run@fullhead}{\hrule}{\hrule\vskip2pt\hrule height 2pt}{}{}
\makeatother
\pagestyle{headandfoot}
%\runningheadrule
\firstpageheadrule
\firstpageheader {\escuela \\ Curso: \curso}
                 {MATEMÁTICA \\ \tpnombre}
                  {\lugar \\ \today}  
%\runningheader{\includegraphics[width=0.1\textwidth]{logo.png}}
%{Matemática:Trabajo Práctico}
%{\today}
\firstpagefooter{Prof: \textit{Lucas H.Trejo}}{}{\thepage}
\runningfooter{Prof: \textit{Lucas H.Trejo}}{}{Pag. \thepage\ of \numpages}

%%%%%%%%%%%%%%%%%%%%%%%%%%%%%%%%%%%%%%%%%%%%%%%%%%%%%%%%%%%%%%%%%%%%%%%%%%%%%%%
                                       %%% REEMPLAZAR LOS PARÁMETROS AQUÍ    %%  
\newcommand{\escuela}{EES Nº 32}                                             %%             
\newcommand{\curso}{  5º 3ª}                                                  %%             
\newcommand{\tpnombre}{Funciones polinómicas}                     %%
\newcommand{\lugar}{La Plata}                                               %%
%%%%%%%%%%%%%%%%%%%%%%%%%%%%%%%%%%%%%%%%%%%%%%%%%%%%%%%%%%%%%%%%%%%%%%%%%%%%%%%


\begin{document}


\begin{questions}
    
    \question Investigar:
    \begin{parts}         
      
        \part  ¿ Cuáles son las características de los gráficos de las funciones polinómicas?
        \part  ¿Qué usos prácticos tienen las funciones polinómicas?
         
    \end{parts}
    
    \question Con ayuda de Geogebra u otro software matemático, dibujar una representación grafica aproximada
    para cada función:
    \begin{parts}
        
    
    \part \( f(x)=(x-2)(x+3) \)
    \part  \( f(x)=(x+3)(2x-1)(x-2) \)
    \part   \( f_ {(x)} $ = $ (x-2)^ {2}   (x+1)^ {2} \) 
    \part   \( f_ {(x)} $ = $ (x+1)^ {3}  (x-2) \)
\end{parts}
   
    \question Para cada una de las funciones del ejercicio anterior, indicar:
    \begin{parts}
        
    \part Grado de la función.    
    \part   Raíces y orden de multiplicidad de cada una de ellas.
    \part   Punto de intersección con el eje $y$.
    \part  Conjunto de positividad,  $C^+$.
    \part  Conjunto de negatividad, $C^-$. 
\end{parts}

\end{questions}
 


 
 %  \begin{figure}[h]
 %      \centering
 %      \includegraphics[width=1 \linewidth]{figura.png}     
 %  \end{figure}

 

\end{document}
